\documentclass{article}
\usepackage[utf8]{inputenc}

\usepackage{amsmath}
\usepackage{amsfonts}
\usepackage{mathtools}
\usepackage{amsthm}
\usepackage{amssymb}

\newtheorem{theorem}{Theorem}
\newtheorem{lemma}{Lemma}
\newtheorem{corollary}{Corollary}

\title{Integral Power Series}
\author{Drew Duncan}
\date{\today}

\begin{document}

\maketitle

\section{Introduction}

This is about the relationship between $\mathbb{Z}[[x]]$, the ring of formal power series with integer coefficients and $\mathbb{Z}_p$, the ring of p-adic integers.  I will show that certain quotients of $\mathbb{Z}[[x]]$ are isomorphic to certain extensions of $\mathbb{Z}_p$.

The following result is in Neukirch, page 114.  Perhaps some illuminating results in the exercises.

\begin{theorem}\label{Z_p}
$\mathbb{Z}[[x]]/(p-x) \cong \mathbb{Z}_p$
\end{theorem}
\begin{proof}
Let the homomorphism $\varphi : \mathbb{Z}[[x]] \rightarrow \mathbb{Z}_p$ be given by $\varphi(x) = p$.  From the unique canonical representation of a p-adic integer $\alpha = \alpha_0 + \alpha_1 p  + \alpha_2 p^2 + \ldots$, with $\alpha_i \in \{0, 1, \ldots, p-1\}$, it is clear that $\varphi$ is surjective.  Further, zero has the unique canonical  representation with $\alpha_i = 0$.  For any $a(x) \in \mathbb{Z}[[x]]$, $a(x) = a_0 + a_1 x + a_2 x^2 + \ldots$, choose $b_1$ so that $a_0 - pb_1 \in \{0, 1, \ldots, p-1\}$.  Thus $a(x) - b_1(p-x) = (a_0 - pb_1) + (a_1 + b_1)x + a_2 x^2 + \ldots = a_0' + (a_1 + b_1)x + a_2 x^2 + \ldots$.  Choosing $b_2$ so that $(a_1 + b_1) - pb_2 \in \{0, 1, \ldots, p-1\}$, we get $a(x) - b_1(p-x) - b_2x(p-x) = a_0' + a_1'x + a_2 x^2 + \ldots$.  Continuing in this way, we get $a'(x)= a_0' + a_1'x + a_2' x^2 + \ldots = a(x) - b(x)(p-x)$ with $b(x) = b_1 + b_2x + b_3x^2 + \ldots$ and $a_i' \in \{0,1,\ldots,p-1\}$.  Thus if $\varphi(a) = 0$, then $a(x)$ differs from zero by a multiple of $(p-x)$.
\end{proof}

First stating Hensel's Lemma as it's stated in Neukirch

\begin{theorem}[Hensel's Lemma]
If a primitive polynomial $f(x) \in \mathcal{O}[x]$ admits modulo $\mathfrak{p}$ a factorization
$$f(x) \equiv \bar{g}(x)\bar{h}(x) \bmod{\mathfrak{p}}$$
into relatively prime polynomials $\bar{g},\bar{h} \in \kappa[x]$, then $f(x)$ admits a factorization
$$f(x) = g(x)h(x)$$
into polynomials $g,h \in \mathcal{O}[x]$ such that $\deg(g) = \deg(\bar{g})$ and
$$g(x) \equiv \bar{g}(x) \bmod{\mathfrak{p}} \mbox{ and } h(x) \equiv \bar{h}(x) \bmod{\mathfrak{p}}.$$
\end{theorem}
\begin{proof}
Let $d = \deg(f)$, $m = \deg(\bar{g})$, hence $d - m \ge \deg(\bar{h})$. Let $g_0, h_0 \in \mathcal{O}[x]$ be polynomials such that $g_0 \equiv \bar{g} \bmod \mathfrak{p}$, $h_0 \equiv \bar{h} \bmod \mathfrak{p}$ and $\deg(g_0) = m$, $\deg(h_0) \le d-m$. Since $(
\bar{g}, \bar{h}) = 1$, there exist polynomials $a(x), b(x) \in \mathcal{O}[x]$ satisfying $ag_0 + bh_0 \equiv 1 \bmod \mathfrak{p}$.  Among the coefficients of the two polynomials $f-g_0h_0 \mbox{ and } ag_0 + bh_0 - 1 \in \mathfrak{p}[x]$ we pick one with minimum value and call it $\pi$.

Let us look for polynomials $g$ and $h$ in the form:

$$g = g_0 + p_1\pi + p_2\pi^2 + \ldots,$$
$$h = h_0 + q_1\pi + q_2\pi^2 + \ldots,$$
where $p_i, q_i \in \mathcal{O}[x]$ are polynomials of degree $< m$, resp. $\le d - m$.
\end{proof}

A brief reminder of the version of Hensel's Lemma we will use ahead.

\begin{theorem}[Hensel's Lemma]
If a primitive polynomial $f(x) \in \mathbb{Z}_p[x]$ admits a factorization modulo $p$ $f(x) = \bar{g}(x)\bar{h}(x) \pmod{p}$ into relatively prime polynomials $\bar{g},\bar{h} \in \mathbb{Z}_p[x]$, then $f(x)$ admits a factorization $f(x) = g(x)h(x)$ into polynomials $g,h \in \mathbb{Z}_p[x]$ with $g(x) \equiv \bar{g}(x) \pmod{p}$, $h(x) \equiv \bar{h}(x) \pmod{p}$, and $deg(g) = deg(\bar{g})$.

\end{theorem}

See \cite{MR1697859} for a proof.

\begin{theorem} \label{factor}
Let $f(x) = p + x^e g(x) \in \mathbb{Z}[x]$, with $p$ a prime, $e \ge 1$, and $g(x)$ a polynomial with constant coefficient not divisible by $p$.  Then $f(x)$ factors as $f(x) = r(x)s(x)$, $r,s \in \mathbb{Z}_p[x]$, $deg(r) = e$, $r_0, \ldots, r_{e-1}$ divisible by $p$, $r_0$ not divisible by $p^2$, $r_e$ and $s_0$ units in $\mathbb{Z}_p$.
\end{theorem}

\begin{proof}
$f(x) \equiv x^e g(x) \pmod{p}$, and by hypothesis $g(x)$ is not divisible by $x$ modulo $p$.  Thus by Hensel's Lemma we get a factorization $f(x) = r(x)s(x)$ with $deg(r) = e$.  Because $r(x) \equiv x^e \pmod{p}$, we must have that $r_0, \ldots, r_{e-1}$ are divisible by $p$ and $r_e$ not divisible by $p$.  Further, because $f_0 = r_0 s_0 = p$, $p^2$ does not divide $r_0$ and $s_0$ is a unit.
\end{proof}

\begin{theorem}\label{expand}
Let $f(x) \in \mathbb{Z}[[x]]$ with $f_0 = p$.  Then $\mathbb{Z}[[x]]/(f) \cong \mathbb{Z}_p[[x]]/(f)$.
\end{theorem}

\begin{proof}
Map $\mathbb{Z}[[x]]$ into $\mathbb{Z}_p[[x]]$ by the obvious inclusion.  Using the same sort of rewriting procedure as in Theorem \ref{Z_p}, every element in $\mathbb{Z}_p[[x]]$ differs from element in the inclusion by a multiple of $f$.  Again, by this rewriting procedure, anything in $\mathbb{Z}_p[[x]]$ that maps to zero in $\mathbb{Z}_p[[x]]/(f)$ differs from 0 by a multiple of (f).
\end{proof}

\begin{theorem}
Let $f(x) \in \mathbb{Z}[x]$ be $f(x) = p + x^e g(x)$ with $g_0$ not divisible by p.  Then $\mathbb{Z}[[x]]/(f)$ is a totally ramified extension of $\mathbb{Z}_p$ of degree $e$.
\end{theorem}

\begin{proof}
By Theorem \ref{factor}, $f(x) = r(x)s(x)$ with $deg(r) = e$ and $s_0$ a unit in $\mathbb{Z}_p$.  Thus $s(x)$ is a unit in $\mathbb{Z}_p[[x]]$.  Therefore, by Theorem \ref{expand}, $\mathbb{Z}[[x]]/(f) \cong \mathbb{Z}_p[[x]]/(f) \cong \mathbb{Z}_p[[x]]/(r)$.  If $\alpha$ is a root of $r(x)$, then $v_p(\alpha) = 1/e$.
\end{proof}

\bibliographystyle{unsrt}
\bibliography{biblio}
\end{document}
